\begin{center}
\thispagestyle{empty}
\vspace{2cm}
\LARGE{\textbf{RESUMEN}}\\[1.0cm]
\end{center}
\thispagestyle{empty}
Este documento presenta el informe del Sprint 1 del proyecto de plataforma digital para la comunidad de Tejarcillos, Alajuelita, promovido por Komuness CR (organización derivada de Coopesinergía). Este primer producto mínimo viable (MVP) se enfoca en reducir la brecha social y tecnológica mediante la entrega de una plataforma estable y funcional que permita a jóvenes y vecinos visibilizar sus talentos, productos y servicios culturales.

El Sprint 1 aborda la continuación del desarrollo parcial existente, mejorando significativamente la estabilidad del sistema, corrigiendo fallos críticos y optimizando la usabilidad. Las funcionalidades implementadas en esta iteración incluyen: sistema de categorización de contenidos, calendario interactivo de eventos, servidor confiable para almacenamiento de archivos, y correcciones integrales en los módulos de publicaciones y biblioteca digital.

Este primer MVP establece las bases tecnológicas sólidas para las futuras iteraciones que incorporarán roles de usuario, perfiles públicos, banco de profesionales y sistema de membresías premium. La iniciativa busca fortalecer el empoderamiento juvenil, la inclusión tecnológica y la participación cultural, consolidándose como un recurso duradero para la comunidad, con una plataforma probada y estable desde su primera versión funcional.